\documentclass[conference,compsoc,final,a4paper]{IEEEtran}

%% Bitte legen Sie hier den Titel und den Autor der Arbeit fest
\newcommand{\autoren}[0]{Wittmann, Yan}
\newcommand{\dokumententitel}[0]{Dashboardgestaltung auf Mobilgeräten}

% Hier muss normalerweise nichts angepasst werden
\usepackage[utf8]{inputenx}
\usepackage[pdftex]{graphicx}
\DeclareGraphicsExtensions{.pdf,.jpeg,.jpg,.png}
\usepackage[cmex10]{amsmath}
\usepackage{algorithmic}
\usepackage{array}
\usepackage{url}
\usepackage[autostyle=true,german=quotes]{csquotes}
\usepackage[
    sorting=none,   % Keine Sortierung
    doi=true,       % DOI anzeigen
    isbn=false,     % ISBN nicht anzeigen
    url=true,       % URLs anzeigen
    maxnames=4,     % Ab 4 Autoren et al. verwenden
    minnames=1,     % und nur den ersten Autor angeben
    style=ieee,
    sorting=nyt]{biblatex}
\usepackage{booktabs}
\usepackage{xcolor}
\usepackage{listings}             % Source Code listings
\usepackage[printonlyused]{acronym}
\usepackage{fancyvrb}
\usepackage{tocloft} % Schönere Inhaltsverzeichnisse

% ausrücken von Auflistungen
\usepackage{enumitem}
\setitemize{leftmargin=*}

%\usepackage[ngerman]{betababel}
\usepackage[german]{babel}

%\DefineBibliographyStrings{ngerman}{
\DefineBibliographyStrings{german}{
    andothers = {{et al\adddot}},  % Immer et al. sagen, auch bei Deutsch als Sprache
}
\usepackage[
    unicode=true,
    hypertexnames=false,
    colorlinks=true,
    linkcolor=black,
    citecolor=black,
    urlcolor=black,
    pdfpagelabels,
]{hyperref}

\hypersetup{
    pdftitle={\dokumententitel},
    pdfauthor={\autoren},
    pdfdisplaydoctitle=true
}

% Makros für typographisch korrekte Abkürzungen
\newcommand{\zb}[0]{z.\,B.}
\newcommand{\dahe}[0]{d.\,h.}
\newcommand{\ua}[0]{u.\,a.}
 % Weitere Einstellungen aus einer anderen Datei lesen

% Literatur einbinden
\addbibresource{literatur.bib}

\begin{document}

% Titel des Dokuments
    \title{\dokumententitel}

% Namen der Autoren
    \author{
        \IEEEauthorblockN{\autoren}
        \IEEEauthorblockA{
            Hochschule Mannheim\\
            Fakultät für Informatik\\
            Paul-Wittsack-Str. 10,
            68163 Mannheim
        }
    }

% Titel erzeugen
    \maketitle
    \thispagestyle{plain}
    \pagestyle{plain}

% Eigentliches Dokument beginnt hier
% ----------------------------------------------------------------------------------------------------------

% Kurze Zusammenfassung des Dokuments
    \begin{abstract}
        Abstract
    \end{abstract}

% Inhaltsverzeichnis erzeugen
    \small\tableofcontents

% Abschnitte mit \section, Unterabschnitte mit \subsection und
% Unterunterabschnitte mit \subsubsection
% -------------------------------------------------------


    \section{Charakterisierung und Motivation für digitale Dashboards}
    % Charakterisierung, Motivation und Anwendungsbereiche


% -------------------------------------------------------


    \section{Anforderungen verschiedener Gerätetypen}

    \subsection{Eigenschaften und Einschränkungen verschiedener Gerätetypen}
    % mobiler Geräte, Desktops und Laptops (Bildschirmgröße, Auflösung, Eingabemethoden, Performance?)

    \subsection{Benutzererwartungen auf verschiedenen Gerätetypen}
    % Unterschiedliche Nutzungsszenarien, Erwartungen und Anforderungen


% -------------------------------------------------------


    \section{Responsive Webdesign (RWD) als Grundlage}

    \subsection{Prinzipien und Ziele}
    % flexible Layouts, Media Queries

    \subsection{RWD durch Browser-Technologien und Frameworks}
    % CSS, JavaScript, Bootstrap


% -------------------------------------------------------


    \section{Design Patterns für digitale Dashboards}

    \subsection{Übersicht über Design Patterns}

    \subsection{Design Patterns für Geräte mit großem Bildschirm}

    \subsection{Design Patterns für Geräte mit kleinem Bildschirm}


% -------------------------------------------------------


    \section{Wechselwirkungen zwischen Bildschirmgröße und Informationsmenge}

    \subsection{Einfluss der Bildschirmgröße auf die Informationsdarstellung}

    \subsection{Balancieren des Abstraktionsgrades und visueller Klarheit}


% -------------------------------------------------------


    \section{Fazit und Ausblick}


% --------------------------------------------------------------------
    \section*{Abkürzungen}
    \addcontentsline{toc}{section}{Abkürzungen}

% Die längste Abkürzung wird in die eckigen Klammern
% bei \begin{acronym} geschrieben, um einen hässlichen
% Umbruch zu verhindern
% Sie müssen die Abkürzungen selbst alphabetisch sortieren!
    \begin{acronym}[IEEE]
        %\acro{A2A}{Application-to-Application}
    \end{acronym}

% Literaturverzeichnis
    \addcontentsline{toc}{section}{Literatur}
    \AtNextBibliography{\small}
    \printbibliography

\end{document}
