\documentclass[conference,compsoc,final,a4paper]{IEEEtran}

%% Bitte legen Sie hier den Titel und den Autor der Arbeit fest
\newcommand{\autoren}[0]{Wittmann, Yan}
\newcommand{\dokumententitel}[0]{Digitale Dashboardgestaltung für Mobilgeräte}

% Hier muss normalerweise nichts angepasst werden
\usepackage[utf8]{inputenx}
\usepackage[pdftex]{graphicx}
\DeclareGraphicsExtensions{.pdf,.jpeg,.jpg,.png}
\usepackage[cmex10]{amsmath}
\usepackage{algorithmic}
\usepackage{array}
\usepackage{url}
\usepackage[autostyle=true,german=quotes]{csquotes}
\usepackage[
            backref,
            sorting=none,   % Keine Sortierung
            doi=true,       % DOI anzeigen
            isbn=false,     % ISBN nicht anzeigen
            url=true,       % URLs anzeigen
            maxnames=6,     % Ab 6 Autoren et al. verwenden
            minnames=1,     % und nur den ersten Autor angeben
            style=ieee,
            sorting=nyt]{biblatex}
\usepackage{booktabs}
\usepackage{xcolor}
\usepackage{listings}             % Source Code listings
\usepackage[printonlyused]{acronym}
\usepackage{fancyvrb}
\usepackage{tocloft} % Schönere Inhaltsverzeichnisse

%\usepackage[ngerman]{betababel}
\usepackage[german]{babel}

%\DefineBibliographyStrings{ngerman}{
\DefineBibliographyStrings{german}{
    andothers = {{et al\adddot}},  % Immer et al. sagen, auch bei Deutsch als Sprache
}
\usepackage[
      unicode=true,
      hypertexnames=false,
      colorlinks=true,
      linkcolor=black,
      citecolor=black,
      urlcolor=black,
      pdfpagelabels,
   ]{hyperref}

\hypersetup{
    pdftitle={\dokumententitel},
    pdfauthor={\autoren},
    pdfdisplaydoctitle=true
}

% Makros für typographisch korrekte Abkürzungen
\newcommand{\zb}[0]{z.\,B.}
\newcommand{\dahe}[0]{d.\,h.}
\newcommand{\ua}[0]{u.\,a.}


 % Weitere Einstellungen aus einer anderen Datei lesen

% Literatur einbinden
\addbibresource{literatur.bib}

\begin{document}

% Titel des Dokuments
    \title{\dokumententitel}

% Namen der Autoren
    \author{
        \IEEEauthorblockN{\autoren}
        \IEEEauthorblockA{
            Hochschule Mannheim\\
            Fakultät für Informatik\\
            Paul-Wittsack-Str. 10,
            68163 Mannheim
        }
    }

% Titel erzeugen
    \maketitle
    \thispagestyle{plain}
    \pagestyle{plain}

% Eigentliches Dokument beginnt hier
% ----------------------------------------------------------------------------------------------------------

% Kurze Zusammenfassung des Dokuments
    \begin{abstract}
        Abstract
    \end{abstract}

% Inhaltsverzeichnis erzeugen
    \small\tableofcontents

% Abschnitte mit \section, Unterabschnitte mit \subsection und
% Unterunterabschnitte mit \subsubsection
% -------------------------------------------------------

    \section{Einleitung}\label{sec:einleitung}

% Überschrift muss man sich selbst überlegen, ein Titel der Lust auf das Lesen des Papers macht (+ Apekte zur Beantwortung der Frage)
% Keine Fragen in Überschriften, auch in der Paperüberschrift
% forschungsfrage, warum ist forschungsfrage relevant, was ist aufgabe der arbeit, was ist stand

\subsection{Methodik und Forschungsfrage}\label{subsec:methodik-forschungsfrage}

Die Forschung in dieser Arbeit beschränkt sich auf die Literaturrecherche, da praktische Implementierungen und empirische Studien in diesem Kurs nicht durchführbar sind.

In erster Linie werden wissenschaftliche Arbeiten, Artikel, Bücher, peer--reviewed Journals und Konferenzbeiträge dazu verwendet, aber auch Statistiken und Internetquellen können zur Ergänzung dieser verwendet werden.
Entsprechend dem Thema beziehen diese sich auf digitales Dashboard--Design, insbesondere im Kontext der Nutzbarkeit auf mobilen Geräten, Desktop--Computern und Laptops.
Die zentrale Forschungsfrage dieser Arbeit lautet:

\begin{quote}
    Wie können digitale Dashboards so gestaltet werden, dass sie die unterschiedlichen Anforderungen von mobilen Geräten, Desktop--Computern und Laptops erfüllen?
\end{quote}

Die Forschung wird durch die Analyse spezifischer Design Patterns und die Untersuchung von \acl{RWD} ergänzt.
Dadurch sollen für Entwickler und Designer von digitalen Dashboards Empfehlungen abgeleitet werden, die eine gute Nutzbarkeit auf verschiedenen Geräten gewährleisten.

\subsection{Definition von Dashboards}\label{subsec:dashboard-definition-limits}

Dashboards werden von \autocite{Few.InformationDashboardDesign.2013} als visuelle Darstellungen der wichtigsten Informationen definiert, die benötigt werden, um ein oder mehrere Ziele zu erreichen.
Diese Informationen wurden konsolidiert und auf einem Bildschirm so angeordnet, dass sie auf einen Blick erfasst werden können.

Nach \autocite[S. 44]{Yigitbasioglu.AReviewOfDashboardsInPerformanceManagement.2012} ist es schwer, eine klare Definition von Dashboards zu finden, da Dashboard--Hersteller dabei sich vor allem auf die Features ihrer Produkte konzentrieren und Akademiker in der Forschung die unterschiedlichen Dashboardtypen in ihren einzelnen Entwicklungsphasen analysieren wollen und darauf Wert legen.
Sie schlagen vor, Dashboards als ein interaktives, visuelles Performance--Überwachungstool zu sehen, das alle Informationen an einem Ort sammelt, die nötig sind, eines oder mehrere Ziele zu erreichen und einem Nutzer die Option bieten, die Daten weiter zu erkunden und direkte Aktionen daraus abzuleiten.

Dashboards können durch verschiedene Technologien und Methoden realisiert werden, wie zum Beispiel durch Webtechnologien, mobile Apps oder Desktopanwendungen.
Die in dieser Arbeit betrachteten Dashboards werden ausschließlich im Kontext von Webtechnologien (mit \ac{HTML}, \ac{CSS} und JavaScript) betrachtet.

% Eigenschaften, Einschränkungen und Erwartungen an verschiedene Gerätetypen
% mobiler Geräte, Desktops und Laptops (Bildschirmgröße, Auflösung, Eingabemethoden, Performance?)
% Unterschiedliche Nutzungsszenarien, Erwartungen und Anforderungen

% "Definition von Dashboards und Einschränkungen auf Mobilgeräten"
% ist das hier wirklich noch mal nötig? kommt in den RWD chaptern auch schon vor


% -------------------------------------------------------


    % Responsive Webdesign (RWD) als Grundlage
    \newpage
\
\newpage


\section{Responsive Webdesign (RWD) als Grundlage}\label{sec:responsive-webdesign-rwd-als-grundlage}

\subsection{Gründe und Definition von RWD}\label{subsec:grunde-und-definition-von-rwd}

Eine Statistik aus den Jahren 2016 und 2017 \autocite{statista.GWI.2017} zeigt hervor, dass Ende 2016 der Anteil der mobilen Internetnutzung bei 53\% lag und Mitte 2017 bereits bei 61\%.
Eine ähnliche Erhebung von Statista \autocite{statista.DataReportal.WeAreSocial.Hootsuite.2023} im Jahr 2022 führt diesen Trend hin zu einer zunehmend mobilgeräteorientierten Gesellschaft fort:
90,7\% der Internetzugriffe wurden von Mobilgeräten aus getätigt, während Laptops mit 74,8\% nur die zweithäufigste Option darstellten.

Mobilgeräte sind demnach heutzutage das primäre Mittel, um auf Inhalte im Internet zuzugreifen.
Um zu zeigen, wie wichtig mobilgeräteorientierte Websites für Unternehmen heutzutage sind, führt \autocite[S. 25]{Harmsen.2018} die Umfrage "`What Users Want Most From Mobile Sites Today"' von Google \autocite{Google.WhatUsersWantFromMobile.2012} auf.
Diese zeigt, dass 96\% der Internetnutzer auf Mobilgeräten bereits auf Websites navigiert haben, die nicht für mobile Geräte optimiert sind.
Die Reaktionen auf solche unoptimierten Websites sind oft negativ: 48\% der Nutzer fühlen sich frustriert und suchen sofort nach einer alternativen Seite, und 52\% geben an, dass sie aufgrund dieser Erfahrung weniger wahrscheinlich mit dem betroffenen Unternehmen interagieren werden.
Eine schlechte Erfahrung für Mobilnutzer zu bieten birgt also für Unternehmen das reale Risiko, potentielle Kunden alleine durch ihre Online-Präsenz zu verlieren.

Eine Website mobilfähig zu machen ist allerdings keine triviale Aufgabe.
\autocite[S. 25-33]{Harmsen.2018} führt einige Herausforderungen für Entwickler von Websites:
die unbekannte Bildschirmgröße der Endapplikation, potentiell niedrige Datenübertragungsraten durch mobile Verbindungen, die unterschiedlichen Interaktionsmöglichkeiten (Maus, Touch-Geräte, \ldots) und weitere.

\emph{Responsive Webdesign (RWD)} möchte diese Probleme angehen und steigt aufgrund der Diversifizierung der Endgeräte, mit denen Nutzer auf Inhalte zugreifen können, in seiner Relevanz über die letzten Jahre immer stärker.
RWD wird nach W3Schools \autocite{W3Schools.ResponsiveWebDesign.2024} als die Praxis definiert, Websites sich automatisch an verschiedene Bildschirmgrößen mithilfe von HTML und CSS anpassen zu lassen.
Die Elemente einer Seite sollen damit also automatisch in ihrer Größe und Positionierung \emph{vom Browser} implizit aus den von den Entwicklern aufgestellen Regeln angepasst werden, sodass das Layout für beliebige Bildschirmgrößen noch immer angemessen bleibt.

Ein Problem mit dieser Definition ist, dass sie bereits die zu verwendenden Technologien vorschreibt.
Daher fügt \autocite[S. 28]{Laati.ImplementingResponsiveDesignInIndustrialDashboardEditor.2017} hinzu, dass sich RWD heutzutage nicht mehr nur auf Anpassungen einer Seite durch CSS beziehen, sondern ganz allgemein auf die Fähigkeit eines Entwicklers Webanwendungen zu schreiben, die sich an eine breite Menge von Gerätetypen anpassen können.
Beispielsweise kann so auch JavaScript verwendet werden, um komplexere Anpassungen über Code zu definieren.

Ein Problem mit dieser Definition ist, dass sie die zu verwendenden Technologien bereits vorschreibt.
\autocite[S. 28]{Laati.ImplementingResponsiveDesignInIndustrialDashboardEditor.2017} erweitert diese Definition und legt Wert darauf, dass sich RWD heutzutage nicht mehr nur auf Anpassungen einer Seite durch CSS beschränkt.
Es geht vielmehr um die Fähigkeit eines Entwicklers, Webanwendungen zu schreiben, die sich an eine große Menge von Gerätetypen anpassen können.
So kann beispielsweise auch JavaScript eingesetzt werden, um komplexere Anpassungen der Website über Code zu ermöglichen.

Jehl \autocite{Jehl.ResponsibleResponsiveWebDesign.2014} erweitert das Konzept von RWD um den Begriff des "`Responsible Responsive Designs"', der die Verantwortungsbewusstheit in den Vordergrund stellt.
Dieses Konzept stellt bei der Gestaltung von Webseiten immer das Medium mit den größten Einschränkungen als Referenz ins Zentrum, sodass jede der geplanten Interaktionen auf allen unterstützten Geräten angemessen funktionieren.
Beispielsweise soll eine mobile Verbindung ausreichen, um eine Webseite vollständig zu laden, und dass der begrenzte Bildschirmraum eines kleinen Geräts alle notwendigen Informationen anzeigen kann.

% -----

\subsection{Theorie zur Umsetzung von RWD}

Responsive Web Design (RWD) hat sich also bereits lange als unverzichtbar für die Entwicklung von plattformübergreifenden Websites etabliert.
Im Folgenden werden einige theoretische und technische Maßnahmen aufgeführt, mit denen RWD umgesetzt werden können.
Dabei wird nicht nur die Perspektive der Entwickler betrachtet, sondern auch die der Content-Erzeuger, die die eigentlichen Inhalte einer Website verfassen, wie auf Blogs oder in Hilfeartikeln.

In ihrer rohesten Form besuchen laut Gustafson \autocite[Kap. 1]{Gustafson.AdaptiveWebDesign.2011} Endnutzer eine Website immer nur aufgrund ihrer Text- oder Bildinhalte, also die eigentlichen Inhalte der Website.
Navigation und Menüs sind prinzipiell für Nutzer uninteressant, solange diese grundlegend ihre Zwecke erfüllen.

Mit dieser Erkenntnis stellt Katajisto \autocite[S. 4]{Katajisto.CreatingSupportContent.2015} Empfehlungen für die Herangehensweise an diese Aufgabe auf.
Bevor die Implementierung einer Website beginnen kann, sollte eine Content-Strategie durch beteiligung aller Beteiligten (Entwickler, Content-Erzeuger, \ldots) entwickelt werden.
Diese Strategie soll aus sog\. Content-Modulen bestehen, atomare Inhaltssegmente, die von Entwicklern individuell gestaltet, auf einer Seite angeordnet und je nach Displaygröße modifiziert oder ausgeblendet werden können.

Eine weitere, recht häufige Strategie ist "`Mobile First"' \autocite{Wroblewski.MobileFirst.2009} zu gestalten.
Dieser Ansatz brint Entwickler dazu, sich zuerst auf die wichtigsten Inhalte zu konzentrieren, da unnötige und optionale Elemente auf Mobilgeräten mit kleinen Displays keinen Platz finden werden.
Der Gestaltungsprozess einer Seite sollte daher mit der einfachsten möglichen Lösung beginnen und erst mit zunehmender Bildschirmgröße neue, größere oder kompliziertere Elemente aufnehmen.
Nice-to-know Informationen sind oft interessant, aber nicht relevant, wenn der verfügbare Bildschirmraum begrenzt ist.

Ein weiterer wichtiger Punkt, besonders für Entwickler, ist das "`Single-Sourcing"' \autocite[S. 3--4]{Katajisto.CreatingSupportContent.2015} von Seiteninhalten.
Dies bedeutet, dass die Inhalte für jede Art von Gerät und Bildschirmgröße trotz ihrer Unterschiede dennoch aus dem gleichen Quellmaterial erzeugt und verteilt werden.
Eine solche Reduktion der Duplizierung von Inhalten verringert den Programmier-, Autoren- und Wartungsaufwand, da nur eine einzige Datenquelle gepflegt werden muss.
In Referenz zu den in der Planungsphase erstellten Content-Module können so Inhalte konsistent und effizient erstellt werden, die durch automatisierte Bedingungen und Variablen dynamisch für jeden Nutzer einzeln an seine Anforderungen angepasst werden.

% -----

\subsection{Support für RWD durch Browser-Technologien und Frameworks}
% flexible Layouts, Media Queries
% CSS, JavaScript, Bootstrap

Handys mit Auflösungen, die denen von großen Desktop-Monitoren gleichen, werden immer häufiger.
Im Jahr 2017 wurde das erste kommerzielle Handy mit einer 4K-Auflösung (2160 × 3840 Pixel, 806 ppi Pixeldichte) veröffentlicht \autocite{Wikipedia.SonyXperiaZ5Premium.2024}.
Nach \autocite{Harmsen.2018} führt die Entwicklung hin zu größeren Pixeldichten dazu, dass die direkte Verwendung von Gerätepixeln als logische Pixel nicht mehr praktikabel ist, da die dargestellten Inhalte ansonsten viel zu klein wären, um von Nutzern lesbar zu sein.
Um den Problemen durch die immer weiter wachsenden Displays entgegenzuwirken, wurde die Einheit der CSS-Pixel eingeführt.
Sie abstrahieren die Gerätepixel zu logischen CSS-Pixeln, die beliebige Vielfache eines Gerätepixels einnehmen können.

\autocite{JiangResponsiveWebDesignModeAndApplication.2014} führt als universelle und generell anwendbare Praxis für die korrekte automatische Erkennung der CSS-Pixelskalierung mit minimalem Aufwand das Hinzufügen des folgenden \emph{meta}--Tags in die \emph{head}--Sektion der Seite.

% @formatter:off
\begin{verbatim}
<meta name="viewport"
      content="width=device-width,
      initial-scale=1" />
\end{verbatim}
% @formatter:on

Diese Maßnahme hat als Effekt, dass ein Web-Element auf jedem Gerät die gleiche physikalische Fläche einnimmt, egal welche Pixeldichte ein Gerät hat.

---

rewrite this

Um RWD zu ermöglichen, gibt es nach Fernandez \autocite[S. 3]{MobileWebResponsiveWebdesign.Fernandez.2012} im Wesentlichen zwei relevante Schritte, bei denen Anpassungen am Layout vorgenommen werden müssen.
Der erste Schritt betrifft das Verhalten beim Überschreiten von sog\. "`Breakpoints"'.
Der zweite Schritt bezieht sich auf die Anpassung des Layouts zwischen diesen Breakpoints.

Breakpoints sind definierte Displayauflösungen, meist in der Horizontalen und in Pixeln gemessen, die die große Menge an möglichen Auflösungen in einige wenige, überschaubare Segmente unterteilen.
Zwischen diesen Breakpoints kann das Layout der Seite prinzipiell auf dieselbe Art und Weise behandelt werden, hier ist es möglich dynamische Anpassungen mit minimalen Auswirkungen auf das Layout zu definieren.
An den Breakpoints selbst muss das Layout jedoch wesentlich verändert werden, um weiterhin angemessen dargestellt zu werden.

Die Breakpoint-Aufteilung lässt sich direkt über das Media Queries Modul von CSS implementieren.

---

\autocite{Katajisto.CreatingSupportContent.2015} darf dazu kein Web-Element hartkodierte Layouts mit absoluten Positionen enthalten, wie Pixeleinheiten.



% -------------------------------------------------------


    % Design Patterns für digitale Dashboards
    \newpage


\section{Design Patterns für digitale Dashboards}\label{sec:design-patterns-list}

Allgemein wird unter einem Design Pattern eine bewährte Lösung für ein wiederkehrendes Problem verstanden.
Im Gegensatz zu Frameworks sind Design Patterns keine konkreten Implementierungen und schließen sich nicht gegenseitig aus, was sie Programmiersprachen- und Technologie-unabhängig macht.

Von \citeauthor[S. 2367]{Schulz.DesignSpaceVisualizationTasks.2013} werden zwei Herangehensweisen an einen Gestaltungsprozess für Visualisierungen vorgestellt:

\begin{itemize}
    \item Die Aufgabenstellung und die Eingabedaten bestimmen die Visualisierung.
    Damit wird die Frage in den Vordergrund gerückt, welche Visualisierungen am besten geeignet sind.
    \item Die Eingabedaten und die Visualisierung bestimmen die Aufgabenstellung.
    Hiermit wird untersucht, wie gut gewisse Aufgaben auf einer gegebenen Visualisierung und einem Datensatz verfolgt werden können.
\end{itemize}

Die folgenden Kapitel widmen sich vor allem anhand der ersten Herangehensweise mit der Vorstellung von Design Patterns, die für digitale Dashboards relevant sind.
Hierbei wird von \citeauthor[S. 3--5]{Bach.DashboardDesignPatterns.2023} zwischen Patterns unterschieden, die die Darstellung der eigentlichen Inhalte betreffen, und solchen, die sich auf die Komposition von Inhalten beziehen.

\subsection{Inhaltsgetriebene Design Patterns}\label{subsec:content-design-patterns}

Inhaltsgetriebene Design Patterns beziehen sich auf die Verwendung von unterschiedlichen Elementen, die auf einem Dashboard platziert werden können.
Jedes dieser Elemente kann unterschiedliche Informationen unterschiedlich gut kommunizieren und erfüllt andere Zwecke.
Sie stehen im Gegensatz zu den Kompositionspatterns, die in \autoref{subsec:conposition-design-patterns} vorgestellt werden.

Einige mögliche Interaktionen von Nutzern mit Inhalten von Dashboards werden in der Taxonomie von \citeauthor[S. 1]{Heer.InteractiveDynamicsVisualAnalysis.2012} beschrieben.
Ein Nutzer kann Inhalte auf alternative Arten visualisieren, filtern, sortieren, gänzlich neue Datenformen aus vorhandenen ableiten, navigieren, organisieren und einzelne Datenpunkte auswählen, um weitere Informationen zu erhalten oder diese zu manipulieren.

\citeauthor[S. 25]{MarcusHomannVassilenaBanovaPaulOelbermannHolgerWittgesandHelmutKrcmar.TowardsUserInterfaceComponentsforDashboardApplicationsonSmartphones.2013} stellen weitere häufige Interaktionstypen mit Dashboards fest:
Favorisieren von Inhalten oder Einstellungen, Kommentarfunktionen und die Möglichkeit, Inhalte zu teilen oder zu bei Datenänderungen aktualisieren.

Diese Interaktionen können durch verschiedene inhaltsgetriebene Design Patterns unterstützt werden.

\subsection{Kompositionsbasierte Design Patterns}\label{subsec:conposition-design-patterns}

\subsection{Zusätzliche Möglichkeiten von mobilen Dashboards}\label{subsec:additional-capabilities-of-mobile-dashboards}

% Use "Tutorial: Mobile BI" S.29+
% push/pull notifications, GPS for local data, take photos, easily shareable,



% -------------------------------------------------------

    \section{Designkompromisse zwischen Bildschirmfläche und Informationsmenge}

Die genannten Design Patterns und Prinzipien sind eine gute Grundlage für die Gestaltung von digitalen Dashboards.
Jedoch gibt es oft unweigerliche High--Level Entscheidungen \autocite[S. 6]{Bach.DashboardDesignPatterns.2023}, auf die ein Entwickler nur wenig Einfluss hat, wie die Zielgruppe, die Gerätetypen und die Verwendungsszenarien in denen das Dashboard funktionieren muss.
Darum müssen die Low--Level Entscheidungen von Designern das Beste aus dem verfügbaren Bildschirmraum, der Struktur, dem Layout, den verwendeten Darstellungsformen und den eigentlichen Inhalten machen.

Effektiv bildet ein Dashboard einen großen Datenraum auf einen kleineren ab, dessen Größe von der verfügbaren Bildschirmgröße abhängt.
Ein Entwickler muss nun also entscheiden, welche Informationen auf einem Dashboard nicht oder nur iteilweiseil gezeigt werden sollen.
Die unterschiedlichen Metriken, die hierbei eine Rolle spielen, sind der \emph{Bildschirmraum einer Seite}, die \emph{Anzahl der Seiten}, der \emph{Interaktionsgrad} und das \emph{Abstraktionsniveau der Informationen} \autocite[S. 6]{Bach.DashboardDesignPatterns.2023}.
Die Herausforderung bei der Planung eines Dashboards besteht also darin, diese Metriken zu minimieren.
In einer idealen Welt bräuchten alle Dashboards nur eine Seite, hätten keine Interaktionen und abstrahieren keine Informationen, so könnte ein Nutzer mit nur einem Blick auf das Dashboard alle relevanten Informationen erkennen, ohne mit ihm interagieren zu müssen.
Allerdings sind all diese Ziele voneinander abhängig und können meist nicht gleichzeitig erreicht werden, weshalb Kompromisse gemacht werden müssen.

Zwangsläufig kann also durch eine Reduktion des Bildschirmraums ein Dashboard nur eine niedrigere Informationsmenge führen.
Damit Nutzer durch den Informationsverlust keine fehlerhaften Schlussfolgerungen ziehen, müssen Dashboardelemente so angepasst werden, dass diese direkt offensichtlich werden.
Nach \autocite[S. 463--464]{Kim.DesignPatternTradeOffs.2021} gibt es für die Anpassung von Diagrammen und anderen Visualisierungsoptionen mehrere Möglichkeiten:

\begin{itemize}
    \item Ganze \emph{Datensätze oder Datenfelder} können hinzugefügt, entfernt, ersetzt, aggregiert oder in mehrere aufgeteilt werden.
    \item Die \emph{Darstellungsart} kann geändert werden, indem Achsen transponiert, umbenannt oder in ihrer Bedeutung geändert oder gänzlich andere visuelle Kodierungen verwendet werden.
    \item Die \emph{Interaktionen} können durch alternative Ereignisse gesteuert oder Tooltips als eigenständige Elemente fixiert werden.
    \item Das \emph{Narrativ} kann durch das Entfernen oder Hinzufügen von Hervorhebungen, das Aufteilen oder Entfernen von Panels, das Hinzufügen oder Umschalten von Anmerkungen sowie das Reduzieren von Text angepasst werden.
    \item Die \emph{Referenz- und Layout-Elemente} können durch das Vereinfachen von Labels, das Kombinieren von Labels mit den eigentlichen Daten, Interaktionselemente ausklappbar statt immer sichtbar zu machen, sowie das Anpassen von Tick--Größen verbessert werden.
\end{itemize}

Am einfachsten ist es natürlich immer, die Inhalte, die nicht auf eine Seite passen, auf mehrere Seiten aufzuteilen.
So können die individuellen Darstellungen in den meisten Fällen einfach belassen werden, da sie nur vertikal verschoben, aber nicht weiter in ihrer Größe oder Anordnung verändert werden.
Das ist durch \emph{Overflow} mit \emph{Scrolling}, durch \emph{Paging} mit Navigations--Buttons oder mit \emph{Links} zu gänzlich anderen Seiten möglich.
Allerdings führt \autocite[S. 6--7]{Bach.DashboardDesignPatterns.2023} neben diesen noch weitere Techniken zum Verwalten von fehlenden Informationen, wie das Hinzufügen oder Erweitern von vorhandenen Interaktionen, beispielsweise mit \emph{Tooltips}.
Bei stark limitierten Bildschirmraum muss die \emph{Parametrisierung} oder auch das \emph{Details--on--demand} verwendet werden.
Diese sind zwar bei einer ersten Implementierung aufwendiger, aber leicht auf zukünftige Änderungen anpassbar.
Ein Nutzer muss nun jedoch immer mit angeben, was genau er vom Dashboard sehen möchte.

% old chapters:
% \section{Wechselwirkungen zwischen Bildschirmgröße und Informationsmenge}
% \subsection{Einfluss der Bildschirmgröße auf die Informationsdarstellung}
% \subsection{Anpassungsstrategien für unterschiedliche Bildschirmgrößen}
% or another one:
% \subsection{Abwägungen zwischen Informationsdichte und Benutzerfreundlichkeit}


% -------------------------------------------------------


    \section{Fazit und Ausblick}

Die Forschung dieser Arbeit konzentrierte sich auf die Gestaltung digitaler Dashboards für verschiedene Gerätetypen.
Die Ergebnisse zeigen, dass Responsive Web Design (RWD) eine zentrale Rolle bei der Anpassung von Dashboards an unterschiedliche Bildschirmgrößen spielt.

Design Patterns wurden als wesentliche Werkzeuge zur Verbesserung der Benutzerfreundlichkeit und Effizienz von Dashboards identifiziert.
Inhaltsgetriebene und kompositionsbasierte Design Patterns erwiesen sich als besonders nützlich.
Die Untersuchung unterstreicht, dass Breakpoints und Media Queries wichtige Techniken zur Umsetzung von RWD sind.

Die Implementierung von RWD stellt Entwickler vor Herausforderungen wie unterschiedliche Bildschirmgrößen und Interaktionsmöglichkeiten.
Trotz dieser Komplexität ist die Anwendung von RWD für eine verbesserte Nutzererfahrung unerlässlich.

Empfehlungen umfassen die konsistente und effiziente Gestaltung von Inhalten für verschiedene Geräte.
Die Ergebnisse betonen die Notwendigkeit, Inhalte dynamisch und anpassbar zu gestalten, um den unterschiedlichen Anforderungen gerecht zu werden.

Eine kritische Betrachtung zeigte, dass es trotz der Vorteile von RWD auch Hindernisse gibt, wie z.B. die Komplexität der Implementierung und die Notwendigkeit, verschiedene Geräteanforderungen zu berücksichtigen.

Für die Zukunft wird erwartet, dass die Bedeutung von RWD und Design Patterns weiter zunimmt.
Zukünftige Forschung sollte empirische Untersuchungen und praktische Implementierungen beinhalten, um die vorgeschlagenen Methoden weiter zu validieren.

Praktische Implementierungen und Benutzerstudien könnten wertvolle Erkenntnisse für die Weiterentwicklung von Designstrategien liefern.
Die Integration neuer Technologien wie künstlicher Intelligenz in das Dashboard-Design könnte weitere Verbesserungen ermöglichen.

Zusammenfassend bietet diese Arbeit einen umfassenden Überblick über die Gestaltung digitaler Dashboards und deren Anpassung an unterschiedliche Gerätetypen.


% 1. **Zusammenfassung**:
% - Knappe Zusammenstellung der wichtigsten Ergebnisse
% - Keine neuen Inhalte im gesamten Fazit

% 2. **Beantwortung der Fragestellung**:
% - Fragen aus der Einleitung beantworten
% - Keine unbeantworteten Fragen lassen

% 3. **Kritische Betrachtung**:
% - Gab es Hindernisse?
% - Wie aussagekräftig sind die Ergebnisse?
% - Beispiel: Literaturkritik

% 4. **Ausblick**:
% - Bezug zum Seminar
% - Folgefragen?
% - Was sollte noch erforscht werden?



% --------------------------------------------------------------------
    \section*{Abkürzungen}
    \addcontentsline{toc}{section}{Abkürzungen}

    % Die längste Abkürzung wird in die eckigen Klammern
    % bei \begin{acronym} geschrieben, um einen hässlichen
    % Umbruch zu verhindern
    % Sie müssen die Abkürzungen selbst alphabetisch sortieren!
    \begin{acronym}[HTML]
        \acro{API}{Application Programming Interface}
        \acro{CSS}{Cascading Style Sheets}
        \acro{HTML}{HyperText Markup Language}
        \acro{RWD}{Responsive Web Design}
    \end{acronym}

% Literaturverzeichnis
    \addcontentsline{toc}{section}{Literatur}
    \AtNextBibliography{\small}
    \printbibliography

\end{document}
