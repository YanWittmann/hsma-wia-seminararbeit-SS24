\documentclass[conference,compsoc,final,a4paper]{IEEEtran}

%% Bitte legen Sie hier den Titel und den Autor der Arbeit fest
\newcommand{\autoren}[0]{Wittmann, Yan}
\newcommand{\dokumententitel}[0]{Dashboardgestaltung auf Mobilgeräten}

% Hier muss normalerweise nichts angepasst werden
\usepackage[utf8]{inputenx}
\usepackage[pdftex]{graphicx}
\DeclareGraphicsExtensions{.pdf,.jpeg,.jpg,.png}
\usepackage[cmex10]{amsmath}
\usepackage{algorithmic}
\usepackage{array}
\usepackage{url}
\usepackage[autostyle=true,german=quotes]{csquotes}
\usepackage[
            backref,
            sorting=none,   % Keine Sortierung
            doi=true,       % DOI anzeigen
            isbn=false,     % ISBN nicht anzeigen
            url=true,       % URLs anzeigen
            maxnames=6,     % Ab 6 Autoren et al. verwenden
            minnames=1,     % und nur den ersten Autor angeben
            style=ieee,
            sorting=nyt]{biblatex}
\usepackage{booktabs}
\usepackage{xcolor}
\usepackage{listings}             % Source Code listings
\usepackage[printonlyused]{acronym}
\usepackage{fancyvrb}
\usepackage{tocloft} % Schönere Inhaltsverzeichnisse

%\usepackage[ngerman]{betababel}
\usepackage[german]{babel}

%\DefineBibliographyStrings{ngerman}{
\DefineBibliographyStrings{german}{
    andothers = {{et al\adddot}},  % Immer et al. sagen, auch bei Deutsch als Sprache
}
\usepackage[
      unicode=true,
      hypertexnames=false,
      colorlinks=true,
      linkcolor=black,
      citecolor=black,
      urlcolor=black,
      pdfpagelabels,
   ]{hyperref}

\hypersetup{
    pdftitle={\dokumententitel},
    pdfauthor={\autoren},
    pdfdisplaydoctitle=true
}

% Makros für typographisch korrekte Abkürzungen
\newcommand{\zb}[0]{z.\,B.}
\newcommand{\dahe}[0]{d.\,h.}
\newcommand{\ua}[0]{u.\,a.}


 % Weitere Einstellungen aus einer anderen Datei lesen

% Literatur einbinden
\addbibresource{literatur.bib}

\begin{document}

% Titel des Dokuments
    \title{\dokumententitel}

% Namen der Autoren
    \author{
        \IEEEauthorblockN{\autoren}
        \IEEEauthorblockA{
            Hochschule Mannheim\\
            Fakultät für Informatik\\
            Paul-Wittsack-Str. 10,
            68163 Mannheim
        }
    }

% Titel erzeugen
    \maketitle
    \thispagestyle{plain}
    \pagestyle{plain}

% Eigentliches Dokument beginnt hier
% ----------------------------------------------------------------------------------------------------------

% Kurze Zusammenfassung des Dokuments
    \begin{abstract}
        Abstract
    \end{abstract}

% Inhaltsverzeichnis erzeugen
    \small\tableofcontents

% Abschnitte mit \section, Unterabschnitte mit \subsection und
% Unterunterabschnitte mit \subsubsection
% -------------------------------------------------------


    \section{Charakterisierung und Motivation für digitale Dashboards}
    % Charakterisierung, Motivation und Anwendungsbereiche


% -------------------------------------------------------


    \section{Anforderungen verschiedener Gerätetypen}

    \subsection{Eigenschaften und Einschränkungen verschiedener Gerätetypen}
    % mobiler Geräte, Desktops und Laptops (Bildschirmgröße, Auflösung, Eingabemethoden, Performance?)

    \subsection{Benutzererwartungen auf verschiedenen Gerätetypen}
    % Unterschiedliche Nutzungsszenarien, Erwartungen und Anforderungen


% -------------------------------------------------------


    \section{Responsive Webdesign (RWD) als Grundlage}

    \subsection{Prinzipien und Ziele}
    % flexible Layouts, Media Queries

    \subsection{RWD durch Browser-Technologien und Frameworks}
    % CSS, JavaScript, Bootstrap


% -------------------------------------------------------


    \newpage


\section{Design Patterns für digitale Dashboards}\label{sec:design-patterns-list}

Allgemein wird unter einem Design Pattern eine bewährte Lösung für ein wiederkehrendes Problem verstanden.
Im Gegensatz zu Frameworks sind Design Patterns keine konkreten Implementierungen und schließen sich nicht gegenseitig aus, was sie Programmiersprachen- und Technologie-unabhängig macht.

Von \citeauthor[S. 2367]{Schulz.DesignSpaceVisualizationTasks.2013} werden zwei Herangehensweisen an einen Gestaltungsprozess für Visualisierungen vorgestellt:

\begin{itemize}
    \item Die Aufgabenstellung und die Eingabedaten bestimmen die Visualisierung.
    Damit wird die Frage in den Vordergrund gerückt, welche Visualisierungen am besten geeignet sind.
    \item Die Eingabedaten und die Visualisierung bestimmen die Aufgabenstellung.
    Hiermit wird untersucht, wie gut gewisse Aufgaben auf einer gegebenen Visualisierung und einem Datensatz verfolgt werden können.
\end{itemize}

Die folgenden Kapitel widmen sich vor allem anhand der ersten Herangehensweise mit der Vorstellung von Design Patterns, die für digitale Dashboards relevant sind.
Hierbei wird von \citeauthor[S. 3--5]{Bach.DashboardDesignPatterns.2023} zwischen Patterns unterschieden, die die Darstellung der eigentlichen Inhalte betreffen, und solchen, die sich auf die Komposition von Inhalten beziehen.

\subsection{Inhaltsgetriebene Design Patterns}\label{subsec:content-design-patterns}

Inhaltsgetriebene Design Patterns beziehen sich auf die Verwendung von unterschiedlichen Elementen, die auf einem Dashboard platziert werden können.
Jedes dieser Elemente kann unterschiedliche Informationen unterschiedlich gut kommunizieren und erfüllt andere Zwecke.
Sie stehen im Gegensatz zu den Kompositionspatterns, die in \autoref{subsec:conposition-design-patterns} vorgestellt werden.

Einige mögliche Interaktionen von Nutzern mit Inhalten von Dashboards werden in der Taxonomie von \citeauthor[S. 1]{Heer.InteractiveDynamicsVisualAnalysis.2012} beschrieben.
Ein Nutzer kann Inhalte auf alternative Arten visualisieren, filtern, sortieren, gänzlich neue Datenformen aus vorhandenen ableiten, navigieren, organisieren und einzelne Datenpunkte auswählen, um weitere Informationen zu erhalten oder diese zu manipulieren.

\citeauthor[S. 25]{MarcusHomannVassilenaBanovaPaulOelbermannHolgerWittgesandHelmutKrcmar.TowardsUserInterfaceComponentsforDashboardApplicationsonSmartphones.2013} stellen weitere häufige Interaktionstypen mit Dashboards fest:
Favorisieren von Inhalten oder Einstellungen, Kommentarfunktionen und die Möglichkeit, Inhalte zu teilen oder zu bei Datenänderungen aktualisieren.

Diese Interaktionen können durch verschiedene inhaltsgetriebene Design Patterns unterstützt werden.

\subsection{Kompositionsbasierte Design Patterns}\label{subsec:conposition-design-patterns}

\subsection{Zusätzliche Möglichkeiten von mobilen Dashboards}\label{subsec:additional-capabilities-of-mobile-dashboards}

% Use "Tutorial: Mobile BI" S.29+
% push/pull notifications, GPS for local data, take photos, easily shareable,



% -------------------------------------------------------


    \section{Wechselwirkungen zwischen Bildschirmgröße und Informationsmenge}

    \subsection{Einfluss der Bildschirmgröße auf die Informationsdarstellung}

    \subsection{Anpassungsstrategien für unterschiedliche Bildschirmgrößen}


% -------------------------------------------------------


    \section{Fazit und Ausblick}


% --------------------------------------------------------------------
    \section*{Abkürzungen}
    \addcontentsline{toc}{section}{Abkürzungen}

% Die längste Abkürzung wird in die eckigen Klammern
% bei \begin{acronym} geschrieben, um einen hässlichen
% Umbruch zu verhindern
% Sie müssen die Abkürzungen selbst alphabetisch sortieren!
    \begin{acronym}[IEEE]
        %\acro{A2A}{Application-to-Application}
    \end{acronym}

% Literaturverzeichnis
    \addcontentsline{toc}{section}{Literatur}
    \AtNextBibliography{\small}
    \printbibliography

\end{document}
