\newpage
\
\newpage

\section{Responsive Webdesign (RWD) als Grundlage}

\subsection{Gründe und Definition von RWD}

Die Relevanz von Responsive Webdesign (RWD) nimmt über die letzten Jahre aufgrund einer Diversifizierung der Endgeräte, mit denen Nutzer auf Inhalte zugreifen können, immer stärker zu.

Laut einer Erhebung von Statista \autocite{statista.DataReportal.WeAreSocial.Hootsuite.2023} wurden im Jahr 2022 90,7\% der Internetzugriffe von Mobilgeräten aus getätigt, während Laptops mit 74,8\% die zweithäufigste Option darstellten.
Eine ähnliche Statistik aus den Jahren 2016 und 2017 \autocite{statista.GWI.2017} zeigt einen klaren Trend hin zu einer zunehmend mobilgeräteorientierten Gesellschaft: Ende 2016 lag der Anteil der mobilen Internetnutzung bei 53\%, Mitte 2017 bereits bei 61\%.

Mobilgeräte sind demnach heutzutage das primäre Mittel, um auf Inhalte im Internet zuzugreifen.
\autocite[S. 25]{Harmsen.2018} führt die Umfrage "`What Users Want Most From Mobile Sites Today"' von Google \autocite{Google.WhatUsersWantFromMobile.2012} auf.
Diese zeigt, dass 96\% der Internetnutzer auf Mobilgeräten Websites gefunden haben, die nicht für mobile Geräte optimiert sind.
Die Reaktionen auf solche unoptimierten Websites sind größtenteils negativ: 48\% der Nutzer fühlen sich frustriert und suchen sofort nach einer alternativen Seite, und 52\% geben an, dass sie aufgrund dieser Erfahrung weniger wahrscheinlich mit dem betroffenen Unternehmen interagieren werden.

\autocite[S. 25-33]{Harmsen.2018} Die Herausforderungen für Entwickler von Websites sind also die unbekannte Bildschirmgröße der Endapplikation, potentiell niedrige Datenübertragungsraten durch mobile Verbindungen, die unterschiedlichen Interaktionsmöglichkeiten (Maus, Touch-Geräte, \ldots) und weitere.
RWD wird laut W3Schools \autocite{W3Schools.ResponsiveWebDesign.2024} als die Praxis definiert, Websites mithilfe von HTML und CSS so zu konfigurieren, dass sie sich automatisch an verschiedene Gerätetypen und Bildschirmgrößen anpassen.
Die Elemente einer Seite werden damit automatisch in ihrer Größe und Positionierung vom Browser angepasst, sodass das Layout für beliebige Bildschirmgrößen noch immer angemessen bleibt.

Ein Problem mit dieser Definition ist, dass sie bereits die zu verwendenden Technologien vorschreibt.
Daher fügt \autocite[S. 28]{Laati.ImplementingResponsiveDesignInIndustrialDashboardEditor.2017} hinzu, dass sich RWD heutzutage nicht mehr nur auf Anpassungen einer Seite durch CSS beziehen, sondern ganz allgemein auf die Fähigkeit eines Entwicklers Webanwendungen zu schreiben, die sich an eine breite Menge von Gerätetypen anpassen können.
Beispielsweise kann so auch JavaScript verwendet werden, um komplexere Anpassungen über Code zu definieren.

Eine weitere Definition von Jehl \autocite{Jehl.ResponsibleResponsiveWebDesign.2014} prägt den Begriff des "`Responsible Responsive Designs"', er fügt also den Begriff der Verantwortungsbewusstheit ein.
Demnach sollen zunächst die zu unterstützenden Gerätearten gesammelt und asugewertet werden.
Im nachfolgenden Entwicklungsprozess sollten dann immer die Einschränkungen des Geräts mit den geringsten Fähigkeiten betrachtet werden.

% Aus dieser reduzierten Bildschirmgröße und

\subsection{RWD durch Browser-Technologien und Frameworks}
% flexible Layouts, Media Queries
% CSS, JavaScript, Bootstrap
