\section{Responsive Webdesign (RWD) als Grundlage}

\subsection{Gründe und Definition von RWD}

Die Relevanz von Responsive Webdesign (RWD) nimmt über die letzten Jahre aufgrund einer Diversifizierung der Endgeräte, mit denen Nutzer auf Inhalte zugreifen können, immer stärker zu.

Laut einer Erhebung von Statista \autocite{statista.DataReportal.WeAreSocial.Hootsuite.2023} wurden im Jahr 2022 90,7\% der Internetzugriffe von Mobilgeräten aus getätigt, während Laptops mit 74,8\% die zweithäufigste Option darstellten.
Eine ähnliche Statistik aus den Jahren 2016 und 2017 \autocite{statista.GWI.2017} zeigt einen klaren Trend hin zu einer zunehmend mobilgeräteorientierten Gesellschaft: Ende 2016 lag der Anteil der mobilen Internetnutzung bei 53\%, Mitte 2017 bereits bei 61\%.

Mobilgeräte sind demnach heutzutage das primäre Mittel, um auf Inhalte im Internet zuzugreifen.
\autocite[S. 25]{Harmsen.2018} führt die Umfrage "`What Users Want Most From Mobile Sites Today"' von Google \autocite{Google.WhatUsersWantFromMobile.2012} auf.
Diese zeigt, dass 96\% der Internetnutzer auf Mobilgeräten Websites gefunden haben, die nicht für mobile Geräte optimiert sind.
Die Reaktionen auf solche unoptimierten Websites sind größtenteils negativ: 48\% der Nutzer fühlen sich frustriert und suchen sofort nach einer alternativen Seite, und 52\% geben an, dass sie aufgrund dieser Erfahrung weniger wahrscheinlich mit dem betroffenen Unternehmen interagieren werden.

\subsection{RWD durch Browser-Technologien und Frameworks}
% flexible Layouts, Media Queries
% CSS, JavaScript, Bootstrap
