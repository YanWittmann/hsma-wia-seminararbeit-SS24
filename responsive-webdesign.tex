\section{Responsive Webdesign (RWD) als Grundlage}\label{sec:responsive-webdesign-rwd-als-grundlage}

\subsection{Gründe und Definition von RWD}\label{subsec:grunde-und-definition-von-rwd}

Eine Statistik aus den Jahren 2016 und 2017 \autocite{statista.GWI.2017} zeigt, dass Ende 2016 der Anteil der mobilen Internetnutzung bei 53\% lag und Mitte 2017 bereits bei 61\%.
Eine ähnliche Erhebung von Statista \autocite{statista.DataReportal.WeAreSocial.Hootsuite.2023} im Jahr 2022 führt diesen Trend hin zu einer zunehmend mobilgeräteorientierten Gesellschaft fort:
90,7\% der Internetzugriffe wurden von Mobilgeräten aus getätigt, während Laptops mit 74,8\% nur die zweithäufigste Option darstellten.

Mobilgeräte sind demnach heutzutage das primäre Mittel, um auf Inhalte im Internet zuzugreifen.
Um zu zeigen, wie wichtig mobilgeräteorientierte Websites für Unternehmen heutzutage sind, führt \autocite[S. 25]{Harmsen.2018} die Umfrage "`What Users Want Most From Mobile Sites Today"' von Google \autocite{Google.WhatUsersWantFromMobile.2012} auf.
Diese zeigt, dass 96\% der Internetnutzer auf Mobilgeräten bereits auf Websites navigiert haben, die nicht für mobile Geräte optimiert sind.
Die Reaktionen auf solche unoptimierten Websites sind oft negativ: 48\% der Nutzer fühlen sich frustriert und suchen sofort nach einer alternativen Seite, und 52\% geben an, dass sie aufgrund dieser Erfahrung weniger wahrscheinlich mit dem betroffenen Unternehmen interagieren werden.
Eine schlechte Erfahrung für Mobilnutzer zu bieten birgt also für Unternehmen das reale Risiko, potentielle Kunden alleine durch ihre Online-Präsenz zu verlieren.

Eine Website mobilfähig zu machen ist allerdings keine triviale Aufgabe.
\autocite[S. 25-33]{Harmsen.2018} führt einige Herausforderungen für Entwickler von Websites:
die unbekannte Bildschirmgröße der Endapplikation, potentiell niedrige Datenübertragungsraten durch mobile Verbindungen, die unterschiedlichen Interaktionsmöglichkeiten (Maus, Touch-Geräte, \ldots) und weitere.

\emph{Responsive Webdesign (RWD)} möchte diese Probleme angehen und steigt aufgrund der Diversifizierung der Endgeräte, mit denen Nutzer auf Inhalte zugreifen können, in seiner Relevanz über die letzten Jahre immer stärker.
RWD wird nach W3Schools \autocite{W3Schools.ResponsiveWebDesign.2024} als die Praxis definiert, Websites sich automatisch an verschiedene Bildschirmgrößen mithilfe von HTML und CSS anpassen zu lassen.
Die Elemente einer Seite sollen damit also automatisch in ihrer Größe und Positionierung \emph{vom Browser} implizit aus den von den Entwicklern aufgestellen Regeln angepasst werden, sodass das Layout für beliebige Bildschirmgrößen noch immer angemessen bleibt.

Ein Problem mit dieser Definition ist, dass sie die zu verwendenden Technologien bereits vorschreibt.
\autocite[S. 28]{Laati.ImplementingResponsiveDesignInIndustrialDashboardEditor.2017} erweitert diese Definition und legt Wert darauf, dass sich RWD heutzutage nicht mehr nur auf Anpassungen einer Seite durch CSS beschränkt.
Es geht vielmehr um die Fähigkeit eines Entwicklers, Webanwendungen zu schreiben, die sich an eine große Menge von Gerätetypen anpassen können.
So kann beispielsweise auch JavaScript eingesetzt werden, um komplexere Anpassungen der Website über Code zu ermöglichen.

Jehl \autocite{Jehl.ResponsibleResponsiveWebDesign.2014} erweitert das Konzept von RWD um den Begriff des "`Responsible Responsive Designs"', der die Verantwortungsbewusstheit in den Vordergrund stellt.
Dieses Konzept stellt bei der Gestaltung von Webseiten immer das Medium mit den größten Einschränkungen als Referenz ins Zentrum, sodass jede der geplanten Interaktionen auf allen unterstützten Geräten angemessen funktionieren.
Beispielsweise soll eine mobile Verbindung ausreichen, um eine Webseite vollständig zu laden, und dass der begrenzte Bildschirmraum eines kleinen Geräts alle notwendigen Informationen anzeigen kann.

% -----

\subsection{Theorie zur Umsetzung von RWD}

Responsive Web Design (RWD) hat sich also bereits lange als unverzichtbar für die Entwicklung von plattformübergreifenden Websites etabliert.
Im Folgenden werden einige theoretische und technische Maßnahmen aufgeführt, mit denen RWD umgesetzt werden können.
Dabei wird nicht nur die Perspektive der Entwickler betrachtet, sondern auch die der Content-Erzeuger, die die eigentlichen Inhalte einer Website verfassen, wie auf Blogs oder in Hilfeartikeln.

In ihrer rohesten Form besuchen laut Gustafson \autocite[Kap. 1]{Gustafson.AdaptiveWebDesign.2011} Endnutzer eine Website immer nur aufgrund ihrer Text- oder Bildinhalte, also die eigentlichen Inhalte der Website.
Navigation und Menüs sind prinzipiell für Nutzer uninteressant, solange diese grundlegend ihre Zwecke erfüllen.

Mit dieser Erkenntnis stellt Katajisto \autocite[S. 4]{Katajisto.CreatingSupportContent.2015} Empfehlungen für die Herangehensweise an diese Aufgabe auf.
Bevor die Implementierung einer Website beginnen kann, sollte eine Content-Strategie durch beteiligung aller Beteiligten (Entwickler, Content-Erzeuger, \ldots) entwickelt werden.
Diese Strategie soll aus sog.\ Content-Modulen bestehen, atomare Inhaltssegmente, die von Entwicklern individuell gestaltet, auf einer Seite angeordnet und je nach Displaygröße modifiziert oder ausgeblendet werden können.

Eine weitere, recht häufige Strategie ist "`Mobile First"' \autocite{Wroblewski.MobileFirst.2009} zu gestalten.
Dieser Ansatz brint Entwickler dazu, sich zuerst auf die wichtigsten Inhalte zu konzentrieren, da unnötige und optionale Elemente auf Mobilgeräten mit kleinen Displays keinen Platz finden werden.
Der Gestaltungsprozess einer Seite sollte daher mit der einfachsten möglichen Lösung beginnen und erst mit zunehmender Bildschirmgröße neue, größere oder kompliziertere Elemente aufnehmen.
Nice-to-know Informationen sind oft interessant, aber nicht relevant, wenn der verfügbare Bildschirmraum begrenzt ist.

Ein weiterer wichtiger Punkt, besonders für Entwickler, ist das "`Single-Sourcing"' \autocite[S. 3--4]{Katajisto.CreatingSupportContent.2015} von Seiteninhalten.
Dies bedeutet, dass die Inhalte für jede Art von Gerät und Bildschirmgröße trotz ihrer Unterschiede dennoch aus dem gleichen Quellmaterial erzeugt und verteilt werden.
Eine solche Reduktion der Duplizierung von Inhalten verringert den Programmier-, Autoren- und Wartungsaufwand, da nur eine einzige Datenquelle gepflegt werden muss.
In Referenz zu den in der Planungsphase erstellten Content-Module können so Inhalte konsistent und effizient erstellt werden, die durch automatisierte Bedingungen und Variablen dynamisch für jeden Nutzer einzeln an seine Anforderungen angepasst werden.

% -----

\subsection{Support für RWD durch Browser-Technologien und Frameworks}
% flexible Layouts, Media Queries
% CSS, JavaScript, Bootstrap

Handys mit Auflösungen, die denen von großen Desktop-Monitoren gleichen, werden immer häufiger.
Im Jahr 2017 wurde das erste kommerzielle Handy mit einer 4K-Auflösung (2160 × 3840 Pixel, 806 ppi Pixeldichte) veröffentlicht \autocite{Wikipedia.SonyXperiaZ5Premium.2024}.
Diese Entwicklung hin zu größeren Pixeldichten führt nach \autocite{Harmsen.2018} dazu, dass die direkte Verwendung von Gerätepixeln als logische Pixel nicht mehr praktikabel ist, da die dargestellten Inhalte ansonsten viel zu klein wären, um von Nutzern lesbar zu sein.
Um den Problemen durch die immer weiter wachsenden Displays entgegenzuwirken, wurde in Browsern die Einheit der CSS-Pixel eingeführt.
Sie abstrahieren die Gerätepixel zu logischen CSS-Pixeln, die beliebige Vielfache eines Gerätepixels einnehmen können.

\autocite{JiangResponsiveWebDesignModeAndApplication.2014} führt als universelle und generell anwendbare Praxis für die automatische Erkennung der optimalen CSS-Pixelskalierung mit minimalem Aufwand das Hinzufügen des folgenden \emph{meta}--Tags in die \emph{head}--Sektion der Seite.

% @formatter:off
\begin{verbatim}
<meta name="viewport"
      content="width=device-width,
      initial-scale=1" />
\end{verbatim}
% @formatter:on

Diese Maßnahme hat als Effekt, dass ein Web-Element auf jedem Gerät die gleiche physikalische Fläche einnimmt, egal welche Pixeldichte ein Gerät besitzt.
Mit dem Problem der Pixeldichte gelöst, können nun weitere Maßnahmen für die verbesserte Unterstützung von RWD aufgeführt werden.

Nach \autocite{Katajisto.CreatingSupportContent.2015} darf es auf einer Seite demnach keine hartkodierte Layouts mit absoluten Positionen geben.
Jedes Web-Element muss auf die aktuelle Browserumgebung reagieren können.

Es gibt nach Fernandez \autocite[S. 3]{MobileWebResponsiveWebdesign.Fernandez.2012} im Wesentlichen zwei relevante Schritte, bei denen Anpassungen am Layout vorgenommen werden müssen.
Der erste Schritt betrifft das Verhalten beim Überschreiten von sog.\ "`Breakpoints"'.
Der zweite Schritt bezieht sich auf die Anpassung des Layouts zwischen diesen Breakpoints.

Breakpoints sind definierte Displayauflösungen, die die große Menge an möglichen Auflösungen in einige wenige, überschaubare Segmente unterteilen.
Die Auflösungen werden meist in der Horizontalen und in CSS-Pixeln gemessen, sind also für alle Gerätearten durch den obigen meta--Tag normalisiert.
Zwischen diesen Breakpoints kann das Layout der Seite auf jeweils dieselbe Art und Weise behandelt werden, hier werden also dynamische Anpassungen mit minimalen Auswirkungen auf das Layout definiert.
An den Breakpoints selbst muss das Layout jedoch wesentlich verändert werden, um weiterhin angemessen dargestellt zu werden.
Sie müssen bei der initialen Problemanalyse gefunden werden und werden im weiteren Verlauf verwendet.

Das Konzept der Breakpoint-Aufteilung lässt sich direkt über das Media Queries Modul von CSS implementieren.
\autocite{JiangResponsiveWebDesignModeAndApplication.2014}
Media Queries erlauben Abfragen einer Vielzahl an Browser-Properties, von denen bei der dynamischen Anpassung von Seiten hauptsächlich die Erkennung der Seitenbreite benötigt wird.
Jeder Media Query-Block stellt ein Segment zwischen zwei Breakpoints dar, wobei immer die obere und untere CSS-Pixelgrenze in den \emph{max-width} und \emph{min-width}--Attributen des Blocks geführt wird.

% @formatter:off
\begin{verbatim}
@media only screen and (min-width:768px)
       and (max-width:1024px) { }
\end{verbatim}
% @formatter:on

Diese Blöcke definiert, können Elemente nun innerhalb dieser über CSS mit Styles versehen werden.
Eine für Entwickler einfache Methode ist nach \autocite{JiangResponsiveWebDesignModeAndApplication.2014} für jedes Breakpoint--Segment die Seitenbreite auf eine fixe Breite zu begrenzen.
So kann mit der folgenden CSS-Klasse \emph{html, body \{min-width: 1333px\}} für Geräte die der Media Query \emph{max-width: 1333px} entsprechen ein Layout erzeugt werden, das sich bis zum nächsten niedrigeren oder höheren Breakpoint nicht verformt.

Eine Alternative dazu ist es, die Seitenbreite zwischen Breakpoints nicht statisch zu machen.
Dann werden die Media Queries ausschließlich dazu verwendet, die Anordnung, Größe oder Sichtbarkeit der eigentlichen Web--Elemente zu beeinflussen.
\autocite[S. 33]{Laati.ImplementingResponsiveDesignInIndustrialDashboardEditor.2017} Bei solchen Layouts mit unbekannten Breiten dürfen selbstverständlich keine fixen Größen für Elemente und die Seite selbst vergeben werden, es müssen dynamische Größen wie \emph{\%}, \emph{vh} oder \emph{vw} verwendet werden.
Diese Strategie erzeugt demnach durch die hier erforderliche Betrachtung jeder möglichen Seitenbreite in Pixeln jedoch einen größeren Entwicklungsaufwand.
Ihre Nutzung ist es bei nicht--scrollenden Dashboardlayouts jedoch oft wert, da es immer einen Mangel an Bildschirmraum für die darzustellenden Informationen gibt und so keine Fläche durch einen Rahmen um die Seite verschenkt wird.

Frameworks wie Bootstrap \autocite{Bootstrap.BuildResponsiveSites.2024} erlauben beide Strategien über vordefinierte CSS-Klassen (\emph{row}, \emph{col} und ihre Varianten).
Diese setzen die maximale Breite einer Seite entweder statisch bei gewissen benannten Breakpoints über die oben genannte Strategie oder über ein \emph{flex}--Layout dynamisch für jede Bildschirmbreite.
Die Größen von Elementen auf der Seite wird ebenfalls entsprechend automatisch angepasst.
Zusammen mit anderen vordefinierten Klassen für die Gestaltung von Überschriften, Textelementen, Formularelementen und vielen weiteren erlaubt Bootstrap eine schnelle Entwicklung von Prototypen und produktionsfähigen Dashboards.

\autocite[S. 72]{Harmsen.2018} Für fortgeschrittene Anpassungen mit größeren strukturellen Auswirkungen auf eine Seite oder Änderungen von Textinhalten ist das Einsetzen von JavaScript jedoch unerlässlich.
Um beispielsweise die Anzahl der anzuzeigenden Elemente in einer Liste dynamisch zu ändern, muss JavaScript die Anzahl der Elemente basierend auf Eigenschaften wie der Bildschirmbreite berechnen und die Elemente entsprechend hinzufügen oder entfernen.
Ebenso muss eine Navigationsleiste auf mobilen Geräten oft in ein ausfahrbares Seitenmenü umgewandelt werden, um Platz zu sparen.
Diese Anpassungen müssen über sogenannte Listener auf die Änderung der Bildschirmgröße reagieren und die Seite entsprechend anpassen.
Änderungen dieser Art wären nicht ohne JavaScript möglich.
