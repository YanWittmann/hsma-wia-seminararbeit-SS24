\section{Fazit und Ausblick}\label{sec:fazit-ausblick}

Die Forschung dieser Arbeit konzentrierte sich auf die Möglichkeiten der Gestaltung digitaler Dashboards für verschiedene Gerätetypen.
Als Ausgangspunkt wurde festgestellt, dass die Anforderung an Web-Applikationen, unterschiedlichste Bildschirmgrößen zu unterstützen, heutzutage unerlässlich geworden ist, wenn eine breite Marktabdeckung erreicht werden soll.

Die Ergebnisse der Forschung zeigen, dass \ac{RWD} nicht nur für allgemeine Web-Applikationen die Grundlage der dynamischen Anpassungsfähigkeit an verschiedene Gerätetypen darstellt, sondern diese Regeln auch bei digitalen Dashboards gültig sind.
Dabei werden vor allem \ac{CSS} und JavaScript verwendet, um automatisch regelbasiert Anpassungen auf eine Seite anzuwenden.

Zu diesen im Web allgemein anwendbaren Maßnahmen wurden zusätzlich spezifische Design Patterns und Prinzipien für digitale Dashboards untersucht.
Diese wurden in unterschiedliche Kategorien wie inhaltsgetriebene und kompositionsbasierte eingeteilt, aber auch Patterns für das allgemeine Umgehen mit unterschiedlichen Datentypen wurden behandelt.
Diese Ansätze sollen es Entwicklern und Designern erleichtern, Dashboards für alle Bildschirmgrößen zu entwickeln.

Allerdings nehmen die Design Patterns und \ac{RWD}--Techniken nicht alle Hürden im Entwicklungsprozess.
Insbesondere die gesteigerte Komplexität der Planung und Implementierung, da nun mehrere Versionen einer Seite entwickelt werden müssen, aber auch die erhöhten Ladezeiten durch das Laden von zusätzlichen Inhalten wie Software--Bibliotheken oder Inhalten, die durch \ac{RWD} ausgeblendet werden, stellen Herausforderungen dar.
Zudem hängt die Effektivität der angewandten Methoden stark von den spezifischen Anforderungen und Kontexten der jeweiligen Anwendung ab.

Für die Zukunft wäre interessant, diese Patterns und Prinzipien in realen Anwendungsszenarien an einem oder mehreren unterschiedlichen konkreten Beispielen zu testen, indem eigene Dashboards entwickelt und evaluiert werden.
Mit der Limitierung genommen, nur Literaturrecherche machen zu können, könnte durch experimentelle Implementierungen und empirische Studien die Wirksamkeit und Anwendbarkeit in unterschiedlichen Kontexten überprüft werden.
Darüber hinaus könnte die Integration neuer Technologien wie künstliche Intelligenz und maschinelles Lernen in die Gestaltung von Dashboards untersucht werden.

% 1. **Zusammenfassung**:
% - Knappe Zusammenstellung der wichtigsten Ergebnisse
% - Keine neuen Inhalte im gesamten Fazit

% 2. **Beantwortung der Fragestellung**:
% - Fragen aus der Einleitung beantworten
% - Keine unbeantworteten Fragen lassen

% 3. **Kritische Betrachtung**:
% - Gab es Hindernisse?
% - Wie aussagekräftig sind die Ergebnisse?
% - Beispiel: Literaturkritik

% 4. **Ausblick**:
% - Bezug zum Seminar
% - Folgefragen?
% - Was sollte noch erforscht werden?
