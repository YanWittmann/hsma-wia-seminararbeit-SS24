\section{Fazit und Ausblick}\label{sec:fazit-ausblick}

Die Forschung dieser Arbeit konzentrierte sich auf die Möglichkeiten der Gestaltung digitaler Dashboards für verschiedene Gerätetypen.
Als Ausgangspunkt wurde festgestellt, dass die Anforderung an Web-Applikationen, unterschiedlichste Bildschirmgrößen zu unterstützen, heutzutage unerlässlich geworden ist, wenn eine breite Marktabdeckung erreicht werden soll.

Die Ergebnisse der Forschung zeigen, dass \ac{RWD} nicht nur für allgemeine Web-Applikationen die Grundlage der dynamischen Anpassungsfähigkeit an verschiedene Gerätetypen darstellt, sondern diese Regeln auch bei digitalen Dashboards gültig sind.
Dabei werden vor allem \ac{CSS} und JavaScript verwendet, um automatisch regelbasiert Anpassungen auf eine Seite anzuwenden.

Zu diesen im Web allgemeinen anwendbaren Maßnahmen wurden zusätzlich mit spezifischen Design Patterns

---

Die wichtigste Erkenntnis dieser Arbeit ist, dass durch den Einsatz von \ac{RWD} und die Anwendung von Design Patterns eine flexible und effektive Gestaltung von Dashboards möglich ist, die sowohl auf mobilen Geräten als auch auf Desktops gut funktionieren.
Diese Anpassungsfähigkeit ermöglicht eine konsistente Benutzererfahrung über verschiedene Plattformen hinweg.

Die zentrale Forschungsfrage, wie digitale Dashboards so gestaltet werden können, dass sie die unterschiedlichen Anforderungen von mobilen Geräten, Desktop-Computern und Laptops erfüllen, konnte somit positiv beantwortet werden.
Es wurde aufgezeigt, dass durch den gezielten Einsatz von \ac{RWD}-Techniken und spezifischen Design Patterns eine optimale Darstellung auf verschiedenen Bildschirmgrößen gewährleistet werden kann.


Eine kritische Betrachtung der Arbeit zeigt jedoch auch einige Herausforderungen und Limitationen auf.
Zum einen basiert die Forschung primär auf Literaturrecherchen und theoretischen Analysen, da praktische Implementierungen im Rahmen dieser Arbeit nicht durchgeführt werden konnten.
Dies schränkt die Aussagekraft der Ergebnisse insofern ein, als dass praktische Tests und Nutzungsstudien fehlen.
Darüber hinaus gibt es in der Literatur unterschiedliche Meinungen zur besten Umsetzung von \ac{RWD}, was die Auswahl der richtigen Ansätze erschweren kann.


Für die Zukunft wäre es daher wünschenswert, die in dieser Arbeit gewonnenen Erkenntnisse durch praktische Implementierungen und Nutzungsstudien zu validieren.
Insbesondere die Untersuchung der tatsächlichen Benutzerfreundlichkeit und Performance von Dashboards auf verschiedenen Geräten könnte wertvolle zusätzliche Einblicke bieten.
Darüber hinaus könnten weiterführende Forschungen die Integration neuer Technologien und Trends, wie Progressive Web Apps (PWA) oder Adaptive Design, in die Gestaltung digitaler Dashboards beleuchten.


% 1. **Zusammenfassung**:
% - Knappe Zusammenstellung der wichtigsten Ergebnisse
% - Keine neuen Inhalte im gesamten Fazit

% 2. **Beantwortung der Fragestellung**:
% - Fragen aus der Einleitung beantworten
% - Keine unbeantworteten Fragen lassen

% 3. **Kritische Betrachtung**:
% - Gab es Hindernisse?
% - Wie aussagekräftig sind die Ergebnisse?
% - Beispiel: Literaturkritik

% 4. **Ausblick**:
% - Bezug zum Seminar
% - Folgefragen?
% - Was sollte noch erforscht werden?
