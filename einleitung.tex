\section{Einleitung}\label{sec:einleitung}

% Überschrift muss man sich selbst überlegen, ein Titel der Lust auf das Lesen des Papers macht (+ Apekte zur Beantwortung der Frage)
% Keine Fragen in Überschriften, auch in der Paperüberschrift
% forschungsfrage, warum ist forschungsfrage relevant, was ist aufgabe der arbeit, was ist stand

\subsection{Methodik und Forschungsfrage}\label{subsec:methodik-forschungsfrage}

Die Forschung in dieser Arbeit beschränkt sich auf die Literaturrecherche, da praktische Implementierungen und empirische Studien in diesem Kurs nicht durchführbar sind.

In erster Linie werden wissenschaftliche Arbeiten, Artikel, Bücher, peer--reviewed Journals und Konferenzbeiträge dazu verwendet, aber auch Statistiken und Internetquellen können zur Ergänzung dieser verwendet werden.
Entsprechend dem Thema beziehen diese sich auf digitales Dashboard--Design, insbesondere im Kontext der Nutzbarkeit auf mobilen Geräten, Desktop--Computern und Laptops.
Die zentrale Forschungsfrage dieser Arbeit lautet:

\begin{quote}
    Wie können digitale Dashboards so gestaltet werden, dass sie die unterschiedlichen Anforderungen von mobilen Geräten, Desktop--Computern und Laptops erfüllen?
\end{quote}

Die Forschung wird durch die Analyse spezifischer Design Patterns und die Untersuchung von \acl{RWD} ergänzt.
Dadurch sollen für Entwickler und Designer von digitalen Dashboards Empfehlungen abgeleitet werden, die eine gute Nutzbarkeit auf verschiedenen Geräten gewährleisten.

\subsection{Definition von Dashboards}\label{subsec:dashboard-definition-limits}

Dashboards werden von \autocite{Few.InformationDashboardDesign.2013} als visuelle Darstellungen der wichtigsten Informationen definiert, die benötigt werden, um ein oder mehrere Ziele zu erreichen.
Diese Informationen wurden konsolidiert und auf einem Bildschirm so angeordnet, dass sie auf einen Blick erfasst werden können.

Nach \autocite[S. 44]{Yigitbasioglu.AReviewOfDashboardsInPerformanceManagement.2012} ist es schwer, eine klare Definition von Dashboards zu finden, da Dashboard--Hersteller dabei sich vor allem auf die Features ihrer Produkte konzentrieren und Akademiker in der Forschung die unterschiedlichen Dashboardtypen in ihren einzelnen Entwicklungsphasen analysieren wollen und darauf Wert legen.
Sie schlagen vor, Dashboards als ein interaktives, visuelles Performance--Überwachungstool zu sehen, das alle Informationen an einem Ort sammelt, die nötig sind, eines oder mehrere Ziele zu erreichen und einem Nutzer die Option bieten, die Daten weiter zu erkunden und direkte Aktionen daraus abzuleiten.

Dashboards können durch verschiedene Technologien und Methoden realisiert werden, wie zum Beispiel durch Webtechnologien, mobile Apps oder Desktopanwendungen.
Die in dieser Arbeit betrachteten Dashboards werden ausschließlich im Kontext von Webtechnologien (mit \ac{HTML}, \ac{CSS} und JavaScript) betrachtet.

% Eigenschaften, Einschränkungen und Erwartungen an verschiedene Gerätetypen
% mobiler Geräte, Desktops und Laptops (Bildschirmgröße, Auflösung, Eingabemethoden, Performance?)
% Unterschiedliche Nutzungsszenarien, Erwartungen und Anforderungen

% "Definition von Dashboards und Einschränkungen auf Mobilgeräten"
% ist das hier wirklich noch mal nötig? kommt in den RWD chaptern auch schon vor
