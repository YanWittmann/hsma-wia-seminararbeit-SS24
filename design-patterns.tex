\newpage


\section{Design Patterns für digitale Dashboards}\label{sec:design-patterns-list}

Allgemein wird unter einem Design Pattern eine bewährte Lösung für ein wiederkehrendes Problem verstanden.
Im Gegensatz zu Frameworks sind Design Patterns keine konkreten Implementierungen und schließen sich nicht gegenseitig aus, was sie Programmiersprachen- und Technologie-unabhängig macht.

Schulz \autocite{Schulz.DesignSpaceVisualizationTasks.2013} stellt zwei Herangehensweisen an einen Gestaltungsprozess für Visualisierungen vor:

\begin{itemize}
    \item Die Aufgabenstellung und die Eingabedaten bestimmen die Visualisierung.
    Damit wird die Frage in den Vordergrund gerückt, welche Visualisierungen am besten geeignet sind.
    \item Die Eingabedaten und die Visualisierung bestimmen die Aufgabenstellung.
    Hiermit wird untersucht, wie gut gewisse Aufgaben auf einer gegebenen Visualisierung und einem Datensatz verfolgt werden können.
\end{itemize}

Die folgenden Kapitel widmen sich vor allem anhand der ersten Herangehensweise mit der Vorstellung von Design Patterns, die für digitale Dashboards relevant sind.
Hierbei wird zwischen Patterns unterschieden, die die eigentlichen Inhalte und deren Darstellung betreffen, und solchen, die sich auf die Komposition von Inhalten beziehen.

\subsection{Inhaltsgetriebene Design Patterns}\label{subsec:content-design-patterns}

\subsection{Kompositionsbasierte Design Patterns}\label{subsec:conposition-design-patterns}

\subsection{Zusätzliche Möglichkeiten von mobilen Dashboards}\label{subsec:zusatzliche-moglichkeiten-von-mobilen-dashboards}

% Use "Tutorial: Mobile BI" S.29+
% push/pull notifications, GPS for local data, take photos, easily shareable,
