\section{Design Patterns für digitale Dashboards}\label{sec:design-patterns-list}

Nach \autocite[S. xi]{Gamma.DesignPatterns.1994} wird unter einem Design Pattern eine bewährte, wiederverwendbare Lösung für ein häufig auftretendes Problem im Softwareentwurf verstanden.
Design Patterns werden entdeckt, und nicht erfunden, und sind somit eine Sammlung von Best Practices, die in der Praxis erprobt und bewährt sind \autocite{Few.InformationDashboardDesign.2013}.
Im Gegensatz zu Frameworks sind Design Patterns keine konkreten Implementierungen und schließen sich nicht gegenseitig aus, was sie Programmiersprachen- und Technologie-unabhängig macht.

Design Patterns gibt es in vielen Bereichen der Softwareentwicklung, hier werden diese für die Gestaltung von Benutzeroberflächen aufgeführt.
Andere Design Patterns können in der Software-Architektur oder in der objektorientierten Programmierung\autocite{Gamma.DesignPatterns.1994} gefunden werden.

Von \autocite[S. 2367]{Schulz.DesignSpaceVisualizationTasks.2013} werden zwei Herangehensweisen an einen Gestaltungsprozess für Visualisierungen vorgestellt:

\begin{itemize}
    \item Die Aufgabenstellung und die Eingabedaten bestimmen die Visualisierung.
    Damit wird die Frage in den Vordergrund gerückt, welche Visualisierungen am besten geeignet sind.
    \item Die Eingabedaten und die Visualisierung bestimmen die Aufgabenstellung.
    Hiermit wird untersucht, wie gut gewisse Aufgaben auf einer gegebenen Visualisierung und einem Datensatz verfolgt werden können.
\end{itemize}

Die folgenden Kapitel widmen sich vor allem anhand der ersten Herangehensweise mit der Vorstellung von Design Patterns, die für digitale Dashboards relevant sind.
Hierbei wird zwischen Patterns unterschieden, die die Darstellung der eigentlichen Inhalte betreffen, und solchen, die sich auf die Komposition von Inhalten beziehen \autocite[S. 3--5]{Bach.DashboardDesignPatterns.2023}.

% -----

\subsection{Inhaltsgetriebene Design Patterns}\label{subsec:content-design-patterns}

Inhaltsgetriebene Design Patterns beziehen sich auf die Verwendung von unterschiedlichen Elementen, die auf einem Dashboard platziert werden können.
Jedes dieser Elemente kann unterschiedliche Informationen unterschiedlich gut kommunizieren und erfüllt andere Zwecke.
Sie stehen im Gegensatz zu den Kompositionspatterns.

Einige mögliche Interaktionen von Nutzern mit Inhalten von Dashboards werden in der Taxonomie von \autocite[S. 1]{Heer.InteractiveDynamicsVisualAnalysis.2012} beschrieben.
Nach dieser können Nutzer Inhalte von Dashboards filtern, sortieren, zwischen Sichten und Seiten navigieren, alternative Darstellungsweisen wählen, gänzlich neue Datentypen durch Kombinieren vorhandener ableiten und einzelne Datenpunkte auswählen, um weitere Informationen zu erhalten oder diese zu manipulieren.
\autocite[S. 25]{MarcusHomannVassilenaBanovaPaulOelbermannHolgerWittgesandHelmutKrcmar.TowardsUserInterfaceComponentsforDashboardApplicationsonSmartphones.2013} stellt weitere häufige Interaktionstypen mit Dashboards fest:
Favorisieren von Inhalten und Einstellungen, Kommentarfunktionen und die Möglichkeit, Inhalte zu teilen oder bei Datenänderungen zu aktualisieren.

Diese Interaktionen werden von durch verschiedene inhaltsgetriebene Design Patterns unterstützt werden.
In den folgenden Unterkapiteln werden Muster, unterschieden in konkrete Visualisierungselemente, Daten--Abstraktionsniveaus und Metainformationen, vorgestellt.

\subsubsection{Visuelle Darstellungs-Patterns}

Die Wahl eines gewünschten Detailgrades der Daten hat einen großen Einfluss darauf, welche der im Folgenden genannten Elemente infrage kommen.
\autocite[S. 3]{Bach.DashboardDesignPatterns.2023} klassifiziert diese auf einer Skala von den detailliertesten Elementen mit den meisten Informationen zu abstrahierenden Elementen, sich eher auf Trends und Muster fokussieren.

\begin{itemize}
    \item \emph{Detaillierte Visualisierungen} sind die Elemente mit den meisten Details.
    Sie enthalten zumeist eine große Menge an Rohdaten und entsprechende Filter/Selektionsoptionen.
    Oft stellen sie einen Kernbestandteil eines Dashboards dar, wenn sie mit genügend Legenden und Erklärungen versehen sind.
    \item \emph{Tabellen und Listen} können je nach Detailgrad Teil der Detailvisualisierungen sein.
    Aufgrund der Kompaktheit der Daten und Übersichtlichkeit durch die tabellarische Anordnung sind sie besonders für den Vergleich von Einzelwerten geeignet.
    Durch Sortier- und Filteroptionen können Nutzer die Daten individuell nach ihren Bedürfnissen anpassen.
    \item \emph{Miniaturdiagramme} sind kleine, kompakte Visualisierungen, die einen schnellen Überblick über Trends geben, ohne konkrete Werte zu zeigen, wie es Sparklines ebenfalls tun \autocite{tufte2006beautiful.2006}.
    Sie können beispielsweise in Tabellen eingebettet werden und stellen in der Regel reduzierte Versionen von Diagrammen dar, die in detailreicheren Visualisierungen an anderen Stellen des Dashboards zu finden sind.
    \item \emph{Fortschrittsbalken und "`Tachos"'} visualisieren Einzelwerte (z.B.\ Prozentsatz) als Balken- oder Kreisdiagramm in einem bekannten Kontext.
    \item \emph{Piktogramme} sind abstrakte Symbole, die Konzepte abhängig vom Kontext eines Dashboards repräsentieren.
    Sie geben die Art, die Existenz oder die Abwesenheit von Daten an, anstatt konkrete Werte zu zeigen.
    \item \emph{Trendpfeile} zeigen die Richtung einer Veränderung in binärer Form (oben, unten) oder als Neigung einer Datenänderung an.
    \item \emph{Zahlen} können alleinstehend ähnlich wie Piktogramme und Trendpfeile verwendet werden, allerdings um konkrete Kennzahlen hervorzuheben.
\end{itemize}

\subsubsection{Dateninformations-Patterns}\label{subsubsec:data-information-patterns}

Diese Patterns stufen nach \autocite[S. 3]{Bach.DashboardDesignPatterns.2023} die Art und den Abstraktionsgrad der in den visuellen Design Pattern präsentieren Informationen ein.

\emph{Detaillierte Datensätze} bieten die umfassendste Sicht auf die Daten und ermöglichen es Nutzern beliebige Werte zu analysieren.
Mit ihnen sind Nutzer am nächsten an den Rohdaten.
Durch \emph{Aggregation} mehrerer Datenpunkte und durch \emph{Filterung} und dem Einführen von \emph{Schwellenwerten} können diese umfassenden Datensätze bewusst zusammengefasst, gefiltert und kritische Bereiche hervorgehoben werden, um spezifische Fragestellungen zu beantworten.
Begleitend zu diesen Patterns gibt es oft interaktive Dashboardelemente, durch die diese Filter und Grenzwerte verändert werden können.
Aus der Datenlage können auch \emph{Werte abgeleitet} werden, die eine durch zusätzliche Informationen angereicherte, transformierte Sicht auf die Daten geben.
Und schließlich können relevante \emph{Einzelwerte} als alleinstehende Elemente hervorgehoben werden, wie z.B.\ der neueste Wert einer Zeitreihe, um komplexe Daten in schnell erfassbare Kennzahlen zu übersetzen.

\subsubsection{Metainformations-Patterns}\label{subsubsec:meta-information-patterns}

\autocite[S. 3]{Bach.DashboardDesignPatterns.2023} Metainformationen stellen einen Kontext und Erklärungen zu den in Dashboards dargestellten Daten bereit, sind aber nicht Teil der Daten selbst.
Oft ergeben sich bereits durch den Kontext des Dashboards gewisse implizite Metainformationen, hier werden jedoch ausschließlich explizite betrachtet.

\emph{Datenquellen} geben an, woher die Daten stammen und schaffen eine verbesserte Nachvollziehbarkeit und Transparenz über die Glaubwürdigkeit der Daten.
Ebenso können \emph{Aktualisierungsinformationen} und \emph{Datenbeschreibungen} in Form von Legenden und Beschreibungstexten die Verständlichkeit und dem Verständnis über die Aktualität der Daten verbessern.
Besondere Datenpunkte oder Entwicklungen können durch \emph{Annotationen} hervorgehoben werden, um Nutzern auf wichtige (eventuell programmatisch erkannte) Aspekte aufmerksam zu machen.

\subsubsection{Datentypen-Patterns}

In der Taxonomie \emph{Type by Task} von \autocite[S. 2--5]{Shneiderman.TheEyesHaveIt.1996} werden sieben verschiedene Datentypen vorgestellt.
Je nach Datentyp können unterschiedliche Visualisierungen und Interaktionen sinnvoll sein.

\begin{itemize}
    \item \emph{1D}:
    Einzelne Datenpunkte, die auf einer Achse oder in einer Liste linear dargestellt werden.
    Die Daten sind in der Regel Code, Textzeilen, einzelne Zahlen oder Kategorienamen.
    \item \emph{2D}:
    Planare Abbildungen oder Zuordnungen zwischen Datenpunkten in einem zweidimensionalen Raum.
    Hierzu zählen beispielsweise Karten und Gebäudepläne, aber auch Tabellen und zweidimensionale Diagramme wie Streudiagramme.
    Diese haben die Möglichkeit, dem Nutzer Gruppierungen und ähnliche Muster zu zeigen.
    \item \emph{3D}:
    Hierzu zählen dreidimensionale Modelle und Diagramme.
    Ein Nutzer kann hier Verhältnisse wie Abstände, oben/unten/links/rechts und Größen erkennen.
    Um zu verhindern, dass Daten hinter weiteren versteckt werden, können Interaktionsmöglichkeiten wie Zoomen und Rotieren angeboten werden oder Transparenzen genutzt werden.
    \item \emph{ND}:
    Die meisten Datentypen, wie relationale Datenbanken oder reale 3D-Modelle lassen sich gut über Visualisierungen in 2D oder 3D darstellen.
    Die Herausforderung beginnt, sobald weitere (n) Dimensionen hinzukommen, denn nun müssen diese hochdimensionalen Daten in einen 2- oder 3-dimensionalen Raum abgebildet werden.
    Wenn n noch recht klein ist, können Knöpfe und Schieberegler genutzt werden, um sich innerhalb der einzelnen Dimensionen zu bewegen.
    Bei höheren Dimensionen können Clusteranalysen und andere Methoden genutzt werden, um die Daten zu reduzieren und zu gruppieren.
    \item \emph{Temporal}:
    Zeitreihen und andere zeitbezogene Daten sind häufig verwendete Datentypen für Messungs- und Beobachtungsdaten.
    Sie unterscheiden sich von den 1D-Daten dadurch, dass sie ein definiertes Zeitintervall mit Start und Ende haben und Datenpunkte sich überlappen können.
    Nutzer können hier Trends über die Zeit erkennen und Werte vor- oder nach einem Referenzpunkt vergleichen.
    \item \emph{Tree/Hierarchical}:
    In hierarchischen Daten sind Datenpunkte in einer Baumstruktur angeordnet, jeder Knoten hat genau einen Elternknoten (außer der Wurzel) und beliebig viele Kindknoten.
    Bäume werden für Klassifikationen und Kategorisierungen von Daten genutzt.
    In einem interaktiven Modell ist es einfacher die Hierarchie zu verstehen, wenn Wurzel- oder Blattknoten fixiert werden können oder Unterbäume ein- und ausgeklappt werden können.
    \item \emph{Network}:
    Netzwerke sind eine spezielle Form von hierarchischen Daten, bei denen die Beziehungen zwischen den Knoten nicht nur hierarchisch, sondern auch zyklisch sein können.
    Diese kommen beispielsweise in sozialen Netzwerken oder in der Logistik vor.
\end{itemize}

% -----

\subsection{Kompositionsbasierte Design Patterns}\label{subsec:conposition-design-patterns}

Es werden nun die Kompositionspatterns vorgestellt.
Diese beschreiben, wie die individuellen Komponenten mit ihren Titeln, Datenelementen und Beschreibungen (übergreifend als Widgets bezeichnet) eines Dashboards angeordnet und gruppiert werden können, um eine sinnvolle Gesamtsicht zu erzeugen.

\subsubsection{Seitenlayout-Patterns}

\autocite[S. 4]{Bach.DashboardDesignPatterns.2023} Seitenlayout-Patterns beschreiben, wie einzelne Widgets auf einem Dashboard angeordnet und gruppiert werden.

\begin{itemize}
    \item \emph{Offene Layouts} platzieren Widgets frei auf einer Seite, meist ohne eine Hierarchie oder Zusammenhänge zwischen einzelnen Widgets vorzugeben.
    Die Widgets sind oft in einem Raster angeordnet, sodass sie sich einfach an unterschiedliche Bildschirmgrößen anpassen lassen und vom Nutzer frei angeordnet werden können.
    \item \emph{Geschichtete Layouts} organisieren Widgets mit einer impliziten Leserichtung in horizontalen oder vertikalen Schichten.
    Je weiter rechts oder unten ein Widget platziert ist, desto detaillierter sind die Informationen, um dem Nutzer zunächst einen Überblick zu geben und dann in die Tiefe zu gehen.
    \item \emph{Tabellenlayouts} ordnen Widgets in einem Raster an, um Zusammenhänge zwischen sich wiederholenden Konzepten zu zeigen.
    \item \emph{Gruppierte Layouts} fassen semantisch zusammengehörige Widgets visuell zusammen, z.B.\ durch gemeinsamen Titel, Rahmen, Hintergrundfarben oder Abstände.
\end{itemize}

\subsubsection{Bildschirmraum-Nutzung-Patterns}

\autocite[S. 4]{Bach.DashboardDesignPatterns.2023} Ein Dashboard kann aus mehreren \emph{Seiten} bestehen, die je nach Pattern unterschiedlich navigiert werden.
Von diesen Seiten ist immer nur eine auf einmal dem Nutzer sichtbar.
Diese Patterns beschreiben die Möglichkeiten, wie diese Seiteninhalte effektiv auf unterschiedlichen Bildschirmgrößen genutzt werden können, aber nicht unbedingt, wie einzelne Seiten zueinander angeordnet sind.

\begin{itemize}
    \item \emph{Bildschirmfüllend}: Der gesamte Inhalt einer Seite passt ohne Interaktion wie Scrollen auf den Bildschirm.
    Dies ist vor allem für statische Übersichtdashboards wichtig, da hier oft keine Interaktionen möglich sind.
    \item \emph{Überlauf}: Einfach zu implementieren, erlaubt ein solches Dashboard durch Scrollen zwischen Seiten zu navigieren.
    \item \emph{Details auf Abruf}: Zusätzliche Inhalte werden z.B.\ durch Tooltips oder modale Popups bei Interaktion eingeblendet.
    \item \emph{Parametrisierung}: Sichtbare Informationen können durch Schieberegler, Checkboxen oder Dropdown-Menüs angepasst werden.
    Dadurch ändert sich die Anzahl der Seiten nicht, aber die Menge an übermittelten Informationen kann dadurch gesteinigt werden.
    \item \emph{Mehrere Seiten}: Inhalte werden auf mehrere getrennte Seiten aufgeteilt, die durch Navigations-Patterns erreichbar sind.
\end{itemize}

\subsubsection{Struktur-Patterns}

\autocite[S. 4]{Bach.DashboardDesignPatterns.2023} führt die folgenden Struktur-Patterns auf, die die Beziehungen zwischen den durch die Bildschirmnutzung entstehenden Seiten eines Dashboards beschreiben.
\autocite{Shneiderman.TheEyesHaveIt.1996} stellt ein Basisprinzip auf, dem alle Designs folgen sollten: die "`Overview first, zoom and filter, then details-on-demand"'-Strategie.
Es sollte also immer mit einer Übersicht begonnen werden, in der durch Interaktionen in die Tiefe gegangen werden kann.

\begin{itemize}
    \item \emph{Einzelseite}: Im einfachsten Fall besteht ein Dashboard aus einer einzigen Seite.
    Hier können weitere Patterns wie Details auf Abruf oder Parametrisierung verwendet werden, um die Menge an Informationen unter Kontrolle zu halten.
    \item \emph{Parallele Strukturen} sind bei sich wiederholenden Datensätzen zu finden.
    Die so verknüpften Seiten sind also inhaltlich und vom Aufbau her ähnlich, wie z.B.\ bei einer Übersicht über verschiedene Standorte eines Unternehmens.
    \item \emph{Hierarchische Strukturen} sind für Drill-Down-Strukturen geeignet, die in jedem Schritt mehr Details zu einem bestimmten Thema zeigen.
    \item \emph{Offene Struktur}: Eine allgemeine Kategorie, die alle anderen Struktur-Patterns umfasst.
\end{itemize}

\subsection{Zusätzliche Möglichkeiten von mobilen Dashboards}\label{subsec:additional-capabilities-of-mobile-dashboards}

% Use "Tutorial: Mobile BI" S.29+
% push/pull notifications, GPS for local data, take photos, easily shareable,

\autocite{Yigitbasioglu.AReviewOfDashboardsInPerformanceManagement.2012}

\autocite{Watson.MobileBi.2015}
